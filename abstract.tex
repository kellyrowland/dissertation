% (This file is included by thesis.tex; you do not latex it by itself.)

\begin{abstract}

Neutral particle radiation transport simulations are critical for radiation shielding and deep
penetration applications. Arriving at a solution for a given response of interest can be 
computationally difficult because of the magnitude of particle attenuation often seen in these 
shielding problems. Hybrid methods, which aim to synergize the individual favorable aspects of
deterministic and stochastic solution methods for solving the steady-state neutron transport
equation, are commonly used in radiation shielding applications to achieve statistically
meaningful results in a reduced amount of computational time and effort. The current state of the
art in hybrid calculations is the Consistent Adjoint-Driven Importance Sampling (CADIS) and 
Forward-Weighted CADIS (\fwc) methods, which generate Monte Carlo variance reduction parameters
based on deterministically-calculated scalar flux solutions. For certain types of radiation
shielding problems, however, results produced using these methods suffer from unphysical
oscillations in scalar flux solutions that are a product of angular discretization. These 
aberrations are termed ``ray effects''.

The Lagrange Discrete Ordinates (LDO) equations retain the formal structure of the traditional
discrete ordinates formulation of the neutron transport equation and mitigate ray effects at high
angular resolution. In this work, the LDO equations have been implemented in the Exnihilo parallel
neutral particle radiation transport framework, with the deterministic scalar flux solutions
passed to the Automated Variance Reduction Generator (ADVANTG) software and the resultant Monte
Carlo variance reduction parameters' efficacy assessed based on results from MCNP5. Studies were 
conducted in both the CADIS and \fwc\ contexts, with the LDO equations' variance reduction
parameters seeing their best performance in the \fwc\ method, especially for photon transport.

\end{abstract}
