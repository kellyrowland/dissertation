\chapter{Introduction}

The work covered in this dissertation includes the implementation of the Lagrange 
Discrete Ordinates (LDO) equations in the Exnihilo parallel neutral particle radiation 
transport framework for the purpose of using the equations' solutions in Monte Carlo
variance reduction parameter generation via the ADVANTG software to improve the results of
simulations run with MCNP5. We start with an analysis of deterministic 
scalar flux results from solving the LDO equations
compared against those of standard discrete ordinates quadrature set types because the 
LDO equations have never before been implemented in a framework such as Exnihilo. Then,
we assess the performance of the Monte Carlo variance reduction parameters generated 
based on the forward and adjoint solutions from the various quadrature set types in the contexts
of both the Consistent Adjoint-Driven Importance Sampling (CADIS) and the Forward-Weighted 
CADIS (\fwc) methods.

\section{Motivation}

Radiation shielding is an important and interesting problem from various perspectives.
Simulation of shielding scenarios is critical for health physics and nuclear security
applications, but arriving at a solution for a given response of interest (e.g., 
neutron flux at a given location) can be computationally difficult in the context of
the magnitude of particle attenuation often seen in shielding problems.

The steady-state neutron transport equation (NTE), introduced below in Section
\ref{sec:nte}, is typically solved using either deterministic methods or stochastic
(Monte Carlo) methods. We will look at each of these solution methods in further 
detail in Chapter \ref{bgch}, but briefly note here that both solution methods have
individual strengths and weaknesses. So-called ``hybrid'' methods aim to combine the
favorable aspects of deterministic and Monte Carlo methods to achieve better results.
Although hybrid methods are used to significant effect in radiation shielding problems,
they do not entirely mitigate the negative aspects of the combined simulation types.

One particular area of study where hybrid methods tend to fall short is in shielding
problems with highly anisotropic particle movement and particle streaming pathways.
This is because the standard implementation of the CADIS and \fwc\ methods is based on scalar
particle flux rather than angular particle flux. So, solutions from deterministic calculations
exclude information about how particles move toward a response of interest.
For problems with strong anisotropies in the particle flux, the importance map and biased source 
developed using the standard space/energy treatment may not represent the real importance well 
enough to sufficiently improve efficiency in the Monte Carlo calculation. 

This work aims to gauge the performance of Monte Carlo biasing parameters based on
scalar flux solutions from solving the LDO equations. We will be employing the LDO equations'
solutions in the standard CADIS and \fwc\ methods to assess how well the LDO representation's
unique treatment of scattering and asymmetry in angle incorporate angular information into the
resultant scalar flux solutions and corresponding Monte Carlo biasing parameters.

\section{Goals and Impacts}
\label{goals}

The primary goal of this work is to assess the forward and adjoint scalar flux 
solutions of the Lagrange Discrete Ordinates equations as input for Monte Carlo 
variance reduction parameter generation in the contexts of the CADIS and \fwc\ methods.
Additional research objectives in support of the primary goal for this work include:

\begin{itemize}
\item{Implement the LDO equations in a neutral particle radiation transport framework 
      designed to solve the traditional discrete ordinates form of the NTE.}
\item{Choose a small variety of test cases in which to assess the various quadrature
      types' deterministic scalar flux solutions for efficacy in Monte Carlo variance reduction 
      parameter generation.}
\item{Compare forward and adjoint scalar flux solutions resultant from the LDO
      equations against those generated with standard discrete ordinates quadrature
      sets for the chosen test scenarios.}
\item{Test the impact of biasing parameters' angular mesh refinement on Monte Carlo results
      across various quadrature types.}
\end{itemize}

\noindent In meeting these objectives as progress towards accomplishing the primary research goal,
we verify the relative accuracy of the deterministic solutions of the LDO equations and then 
examine how they perform as the deterministic solver for hybrid methods. The test problems used 
are those that challenge hybrid methods in general, and so we have generated
a variety of results of interest to the community at large.

\section{The Neutron Transport Equation}
\label{sec:nte}

The way in which neutrons move, known as ``neutron transport", is governed by the
time-dependent neutron transport equation (NTE) \cite{dude}:

\begin{multline}
\frac{1}{v}\frac{\del}{\del t}\psi(\vecr,E,\bo,t) +  \bo \cdot \nabla \psi(\vecr,E,\bo,t) + 
\Sigma_t(\vecr,E) \psi(\vecr,E,\bo,t) =  \\
\int_0^\infty\int_{4\pi} \Sigma_s(\vecr,E'\rightarrow E,\bo'\cdot\bo)
\psi(\vecr,E',\bo',t)d\bo'dE' + Q(\vecr,E,\bo,t),
\label{eq:tnte}
\end{multline}

\noindent where $\vecr$ is the neutron position, $E$ is the energy of the neutron,
$\bo$ is the direction of travel of the neutron, and $t$ is the time. The combination of 
$(\vecr,E,\bo,t)$ is generally referred to as the ``phase space'' of the particles. $\psi$ 
denotes angular neutron flux, $\Sigma$ represents the cross section of a material, and $Q$ is
any additional source (fission, a fixed source, etc.) of neutrons.

We are often interested in situations in which the particle flux is not a function of time. In
these cases, we solve the time-independent (steady-state) neutron transport equation, written as

\begin{multline}
\bo \cdot \nabla \psi(\vecr,E,\bo) + \Sigma_t(\vecr,E) \psi(\vecr,E,\bo) =  \\
\int_0^\infty\int_{4\pi} \Sigma_s(\vecr,E'\rightarrow E,\bo'\cdot\bo)
\psi(\vecr,E',\bo')d\bo'dE' + Q(\vecr,E,\bo).
\label{eq:nte}
\end{multline}

The steady-state neutron transport equation can be thought of as a balance equation in 
which the neutron losses represented on the left-hand side of the equation are equal to
the neutron gains represented on the right-hand side of the equation \cite{dude}. The 
first term on the left-hand side of the steady-state NTE accounts for all neutrons lost by 
streaming out through the surface of the system being considered. The second term in 
the left-hand side of the steady-state NTE accounts for all neutrons 
lost to collisions; this includes neutrons lost via absorption as well as neutrons that
exit the phase space of interest by scattering into a different energy and angle. The 
right-hand side of the equation totals system gains by summing up all neutrons that 
scatter into the phase space of interest from different energies and angles along with 
neutrons created from the source. The scattering term depends not on the individual
angles $\bo$ and $\bo'$ but on their dot product.

The derivative in the first term of the steady-state NTE suggests that we must prescribe 
appropriate boundary conditions for the equation in order to solve the problem. The boundary
conditions depend on the given problem of interest and will be discussed in more detail later.
The following chapter will provide in-depth explanations of various solution methods 
for the time-independent NTE.

\section{Dissertation Outline}

The remainder of this dissertation is structured so as to provide a relevant theoretical
background and discussion as a prelude to the eventual presentation of the results and analysis
arrived at in meeting the sundry and ultimate objectives listed above in Section \ref{goals}.
Chapter \ref{bgch} provides a theoretical basis of the foundation of solution methods for the
NTE, followed by a discussion of pertinent work in the area of hybrid methods. Specific attention
is given to developments that aimed to incorporate angular information into Monte Carlo biasing
parameters; we will see in Section \ref{sec:ldo} that the interest in using the LDO equations'
solutions for Monte Carlo variance reduction parameter generation stems from the unique way in
which the LDO equations treat particle scattering.

Transitioning from theory to practice, Chapter \ref{ch:method} examines the
traditional discrete ordinates equations and the LDO equations from the
perspective of implementing both sets of  equations in a neutral particle
radiation transport software framework. Specific focus is given on the
differences in implementing the contrasting equations; a discussion of details
regarding the solution of the LDO equations in a framework designed to solve the
conventional discrete ordinates equations concludes the chapter.  We note that
the discussion in Chapters \ref{bgch} and \ref{ch:method} focuses on neutron
transport, but the solution methods can be leveraged for photon transport as
well; Chapter \ref{sec:results} presents the test case scenarios examined in
this work with results from the various hybrid methodologies for both neutron
and photon transport problems. Chapter  \ref{ch:conc} concludes the dissertation
with a summary review of the results and  analysis followed by a brief
discussion of future work paths.
